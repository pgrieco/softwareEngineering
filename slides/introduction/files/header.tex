\documentclass{beamer} % Gliederung im Kopf, sections und subsections
\renewcommand{\baselinestretch}{1.3}\normalsize	
\usetheme{default}
\setbeamertemplate{navigation symbols}{} 	
\setbeamertemplate{footline}[frame number]  

%\usepackage{beamerthemesplit}
%\useoutertheme[subsection=false]{smoothbars}
%\usepackage[final]{pdfpages}
\usepackage{bibentry}	
\usepackage{bm}	
\usepackage{graphicx}	
\usepackage{relsize}
\usepackage[round,longnamesfirst]{natbib}
\usepackage{bm}																									%matrix symbol
\usepackage{verbatim}
\usepackage{setspace}																					%Fu�noten (allgm.
\usepackage{hyperref}		
\hypersetup{colorlinks=true,urlcolor=darkblue}														%Zeilenabst�nde)
\usepackage{threeparttable}
\usepackage{bbm}
\usepackage{epstopdf}
\usepackage{lscape}																							%Querformat
\usepackage[latin1]{inputenc}																		%Umlaute
\usepackage{graphicx}
%\graphicspath{{../shared-files/}}
\usepackage{booktabs}
\usepackage{amsmath}
\usepackage{amssymb}
%\usepackage{uarial}

\usepackage{tabularx}
\usepackage{fancybox}																						%Boxen und Rahmen
\usepackage{appendix}
\usepackage{enumerate}
\usepackage{eurofont}	
%EURO Symbol	
\usepackage{tabularx}
\usepackage{longtable}																					%Mehrseitige Tabellen
\usepackage[T1]{fontenc}
\usepackage{color,colortbl}																			%Farbige Tabellen
\usepackage{threeparttable}
\usepackage{hyperref}
\DeclareMathOperator*{\argmin}{arg\,min}
\usepackage{subfigure}
\usepackage{tikz}
\usetikzlibrary{shapes,arrows,positioning,chains,fit,calc,matrix,decorations.pathreplacing}
\usepackage{xcolor}
\usetikzlibrary{intersections}


%\usepackage{cmbright}
\def\newblock{\hskip .11em plus .33em minus .07em}
\newcommand{\N}{\mathbb{N}}
\newcommand{\cov}{\mathrm{cov}\thin}
\newcommand{\thin}{\thinspace}
\newcommand{\thick}{\thickspace}

\newcommand{\vect}[1]{\mathbf{#1}}
\newcommand{\myfrac}[3][0pt]{\genfrac{}{}{}{}{\raisebox{#1}{$#2$}}{\raisebox{-#1}{$#3$}}}
\newcommand{\U}{\mathrm{U}}	%Uniform Distribution
\newcommand{\D}{\mathrm{D}}	%Dirichlet Distribution
\newcommand{\W}{\mathrm{W}}	%Wishart Distribution
\newcommand{\E}{\mbox{E}}		%Expectation
\newcommand{\Prob}{\mbox{Pr}}		%Expectation
\newcommand{\Iden}{\mathbb{I}}	%Identity Matrix
\newcommand{\Ind}{\mathbb{\mathrm{I}}}	%Indicator Function
\newcommand{\argmax}{\mathrm{argmax}\thin}
\newcommand{\Tau}{\mathcal{T}\thin}

\newcommand{\var}{\mathrm{var}\thin}
\newcommand{\plim}{\mathrm{plim}\thin}
\newcommand\indep{\protect\mathpalette{\protect\independenT}{\perp}}
\def\independenT#1#2{\mathrel{\rlap{$#1#2$}\mkern5mu{#1#2}}}
\newcommand{\notindep}{\ensuremath{\perp\!\!\!\!\!\!\diagup\!\!\!\!\!\!\perp}}%

\newcommand{\mc}{\multicolumn}
% weitere Optionen:
% secbar: Gliederung im Kopf, nur sections (alternativ zu subsecbar)
% handout: Produktion von Handouts, keine Animationen
\definecolor{darkblue}{rgb}{0,.35,.62}
\definecolor{lightblue}{rgb}{0.8,0.85,1}
\definecolor{lightgrey}{gray}{0.1}	%Farben mischen

%	kbordermatrix options


\def\firstcircle{[name path=firstcircle] (0,0) circle (3cm)}
\def\secondcircle{[name path=secondcircle] (55:3.11111cm) circle (3cm)} %No idea how the numbers selected work; they were derived from what I found online. They seem to function though!
\def\thirdcircle{[name path=thirdcircle] (0:3.5cm) circle (3cm)}


\definecolor{lightgrey}{gray}{0.90}	%Farben mischen
\definecolor{grey}{gray}{0.85}
\definecolor{darkgrey}{gray}{0.65}
\definecolor{lightblue}{rgb}{0.8,0.85,1}

\renewcommand{\arraystretch}{1.5}

\usepackage{tikz}
\usetikzlibrary{arrows,decorations.pathmorphing,backgrounds,positioning,fit,petri}
\renewcommand*{\familydefault}{\sfdefault}

\tikzset{forestyle/.style = {rectangle, thick, minimum width = 5cm, minimum height = 0.5cm, text width = 4.5cm, outer sep = 1mm},  
  pre/.style={<-, shorten <=1pt, >=stealth, ultra thick},
  extend/.style={<-,dashed, shorten <=1pt, >=stealth, ultra thick}}

\setbeamertemplate{enumerate item}{\textbf{(\theenumi)}}
\setbeamertemplate{enumerate subitem}{(\roman{enumii})}

\makeatletter
\newenvironment{cenumerate}{%
  \enumerate
  \setcounter{\@enumctr}{\csname saved@\@enumctr\endcsname}%
}{%
  \expandafter\xdef\csname saved@\@enumctr\endcsname{\the\value{\@enumctr}}%
  \endenumerate
}
\newenvironment{cenumerate*}{%
  \enumerate
}{%
  \expandafter\xdef\csname saved@\@enumctr\endcsname{\the\value{\@enumctr}}%
  \endenumerate
}
\makeatother

%-------------------------------------------------------------------------------
% Full Justification
%-------------------------------------------------------------------------------
\usepackage{ragged2e}
\usepackage{lipsum}
\makeatletter
\renewcommand{\itemize}[1][]{%
  \beamer@ifempty{#1}{}{\def\beamer@defaultospec{#1}}%
  \ifnum \@itemdepth >2\relax\@toodeep\else
    \advance\@itemdepth\@ne
    \beamer@computepref\@itemdepth% sets \beameritemnestingprefix
    \usebeamerfont{itemize/enumerate \beameritemnestingprefix body}%
    \usebeamercolor[fg]{itemize/enumerate \beameritemnestingprefix body}%
    \usebeamertemplate{itemize/enumerate \beameritemnestingprefix body begin}%
    \list
      {\usebeamertemplate{itemize \beameritemnestingprefix item}}
      {\def\makelabel##1{%
          {%
            \hss\llap{{%
                \usebeamerfont*{itemize \beameritemnestingprefix item}%
                \usebeamercolor[fg]{itemize \beameritemnestingprefix item}##1}}%
          }%
        }%
      }
  \fi%
  \beamer@cramped%
  \justifying% NEW
  %\raggedright% ORIGINAL
  \beamer@firstlineitemizeunskip%
}

\newcommand{\beginbackup}{
   \newcounter{framenumbervorappendix}
   \setcounter{framenumbervorappendix}{\value{framenumber}}
}
\newcommand{\backupend}{
   \addtocounter{framenumbervorappendix}{-\value{framenumber}}
   \addtocounter{framenumber}{\value{framenumbervorappendix}} 
}
